\documentclass[]{extarticle}
\usepackage[filigrane]{piprime}
% 8<------8<------8<------8<------8<------8<------8<------8<------
% \usepackage[nobottomtitles]{titlesec}
\usepackage{titlesec}
% \renewcommand{\thesubsection}{\arabic{subsection}}
\titleformat{\subsection}{\normalfont}{%
  \filright
  \bfseries
}{0pt}{\large\bfseries}
\titlespacing{\subsection}{0pt}{*4}{*0.5}

\newcommand{\codepropale}{20100128CACF}

\newcommand{\bigrule}{\titlerule[0.8pt]}
\newcommand{\colorunderline}[1]{\color{mgold}\underline{\color{mred}#1}}
\titleformat{\section}[block]{%
  \color{mgold}%
  \titlerule[.8pt]%
  \vspace{1pt}%
  \titlerule%
  \filcenter%
  \color{mred}
}{%
}{0pt}{\Large\bfseries\colorunderline}%[\titlerule\vspace{1pt}\bigrule]

\titlespacing{\section}{10pt}{*5}{*2}
% \titleformat{ command }[ shape ]{ format }{ label }{ sep }{ before }[ after ]
% 8<------8<------8<------8<------8<------8<------8<------8<------

% \setlength{\parindent}{0pt}
\SetWatermarkFontSize{50pt}
\SetWatermarkAngle{60}
\SetWatermarkText{\shortstack{\letitre\\[5mm]
    \fontsize{60}{60pt}%
    \selectfont www.piprime.fr}}%

% \SetWatermarkFontSize{80pt}
% \SetWatermarkText{\shortstack{\color{mred}BROUILLON\\[5mm]NON CONTRACTUEL}}

\titreformat{\huge}
\titre{\objet\\[3mm]\client\vspace*{5mm}}
\fancypagestyle{otherpage}{
  \cfoot{\objet\\\client\vspace*{5mm}}
}
\pagestyle{otherpage}

\newcommand{\client}{\textsc{!§!(read-string "Client (le ...): ")!§!}}
\newcommand{\objet}{!§!(read-string "Objet (le ...): ")!§!}
\newcommand{\Leclient}{Le \client\xspace}
\newcommand{\leclient}{le \client\xspace}

\begin{document}
\fairetitre

\section{Demandeur \slash{} Commanditaire}
\leclient

\section{Objet de la proposition}
Création, réalisation, maintenance et hébergement d'\textbf{un site internet} pour \leclient

\section{Document joint à la demande \slash{} \og{}Cahier des charges\fg{}}
Aucun document ou \og{}cahier des charges\fg{} déterminant
précisément les besoins du demandeur n'accompagne la présente
proposition. En conséquence de quoi, \textbf{cette dernière ne représente
  qu'une première ébauche en vue de la définition d'un \og{}cahier des
  charges\fg{} qui permettra l'élaboration d'une proposition officielle}
d'engagement des travaux.
Pour ce faire, nous nous proposons de vous accompagner dans
l'élaboration du cahier des charges décrivant votre demande de
projet. Aussi \textbf{ce document peut servir de support en fonction du
  degré de correspondance entre l'offre que nous développons ci-dessous
  et vos besoins réels finaux}.

\section{Votre demande \slash{} Vos besoins}
\leclient souhaite doter .....  d'un site internet à but collaboratif et communautaire.
Par ce biais, cette dernière entend :
\begin{itemize}
\item \textbf{améliorer et étendre la communication} ..........;
\item \textbf{accroître la collaboration et l'interactivité}  entre .............;
\item \textbf{promouvoir les échanges humains et technologiques} dans un cadre numérique
  ouvert, moderne et novateur;
\item \textbf{valoriser ses atouts et ses compétences} en s'appuyant sur les nouvelles technologies.
\end{itemize}

\section{Notre proposition}\label{sec:les-besoins}
Afin de répondre à votre désir de site internet à but collaboratif, nous vous proposons la mise en place
d'une solution de type \og{}système de gestion de contenu\fg{} ou
\textit{SGC} (\textit{CMS} --~\textit{Content Management Systems}~--
en anglais) basée sur le produit \textbf{\typo}.


\begin{itemize}
\item \textbf{Vous maîtriserez vous-même le contenu de votre site}.\\
  \textbf{Aucune connaissance technique particulière n'est nécessaire} pour alimenter
  et utiliser le site. L'inclusion de texte formaté et d'images se
  fait par l'intermédiaire d'un éditeur intégré puissant tout en étant
  ergonomique et simple d'utilisation. Et cela s'exécute tout aussi
  facilement qu'avec votre éditeur de texte favori.

\item \textbf{Vous aurez à votre disposition un large panel de contenus}.
  \begin{itemize}
  \item \textbf{un système de \news thématiques} avec en
    particuliers les \news de l'école, les \news de
    chaque classe (sortie, information etc) et les \news de
    l'association des parents d'élèves.\\
    Ce système de \news intègre \textbf{un système de
      syndication\footnote{\url{http://fr.wikipedia.org/wiki/Syndication_de_contenu}}}
    \og{}RSS\fg{} afin que chacun puisse être informé en temps réel des
    dernières nouveautés;
  \item \textbf{un calendrier des évènements} avec classification par
    catégories (typiquement une catégorie par classe, une catégorie
    \og{}administration\fg{} et une catégorie \og{}parents
    d'élèves\fg{} mais la classification est extensible à l'infini);
  \item \textbf{un blog par classe}, chaque blog devra pouvoir être administré
    directement par les élèves grâce à une interface très simplifiée;
  \item \textbf{un gestionnaire de documents numériques} pour l'administration,
    pour chaque classe et pour les parents d'élèves;
  \item \textbf{des galeries photos} utilisables dans n'importe quelle page;
  \item \textbf{des vidéos} utilisables dans n'importe quelle page (\textbf{sur
      conditions});
  \item \textbf{des forums thématiques} publics ou privés (réservés aux
    enseignants ou aux parents d'élèves par exemple).
  \end{itemize}
\item \textbf{Vous pourrez travailler à plusieurs personnes sur un même
    groupe de pages}.\\
  Voir la rubrique ci-dessous.
\item \textbf{Vous gérerez finement les droits des utilisateurs et des
    contributeurs}.\\
  Cela comprend:
  \begin{itemize}
  \item \textbf{la possibilité de protéger certaines parties du site} par
    authentification des utilisateurs; Ce qui implique la mise en place d'un
    système automatique d'inscription. Ainsi les parents d'élèves
    peuvent avoir des services et des accès privilégiés qu'un visiteur
    non authentifié n'aura pas.
  \item \textbf{la création de groupes de travail} qui permet de travailler à
    plusieurs sur un même groupe de pages (mise en commun de document,
    échange d'information, modification du calendrier, etc);
  \item la possibilité de créer des contributeurs qui possèdent
    certains droits administratifs;
  \item la présence d'un (ou plusieurs) administrateur qui a le pouvoir de
    modifier l'intégralité du site ainsi que de  modifier les droits
    de tous les contributeurs.
  \end{itemize}
\item \textbf{Vous serez en possession d'un système extensible}.\\
  Cela se traduit concrètement par la possibilité d'\textbf{ajouter des
    fonctionnalités} grâce à l'installation de
  greffons\footnote{\url{http://fr.wikipedia.org/wiki/Plugin}}
  (plugins ou addons). Ce système permet une modernisation et
  une évolution fonctionnelle du site sans nécessité la réfection
  totale de sa structure et de son contenu.

  % \item \textbf{Vous pouvez entièrement restructurer le contenu}.\\
  %   Un administrateur du site préalablement formé peut, en temps
  %   réel, réorganiser l'ensemble de la structure et du contenu du site
  %   sans nécessairement passer par un intermédiaire. Cette possibilité
  %   de réorganisation est indépendante du système d'exploitation
  %   utilisée et elle est possible depuis n'importe quel ordinateur
  %   connecté à l'internet.
\end{itemize}

% \newpage

!§!(when  (yes-or-no-p "Insérer promotion TYPO ?")
(insert "\\section{À propos de \\typo}
\\begin{wrapfigure}{l}{22mm}
  \\centering
  \\includegraphics[width=20mm]{/home/pi/aa_documents/piprime/logo/typo_43.png}
  \\vspace*{-4mm}
\\end{wrapfigure}
Notre solution est basée sur
\\typo\\footnote{\\url{http://www.typo3.org/}} qui dispose de
\\textbf{références prestigieuses} à travers le monde, il est utilisé
aussi bien par des \\textit{PME/PMI}, des collectivités locales que par
des grands groupes.

\\subsection{\\typo est un système de gestion de contenu professionnel}
Conçu pour créer et gérer \\textbf{tous types de sites} internet ou Intranet, il
est sous licence libre (\\og{}open-source\\fg{}); cela lui confère
\\textbf{garantie de pérennité}, \\textbf{flexibilité} et \\textbf{évolutivité} tout en
bénéficiant d'un très grand nombre d'interfaces, de fonctions et de
modules prêts à l'emploi.
\\begin{figure}[h]
  \\centering
  \\includegraphics[width=0.5\\linewidth]{include/pages.png}
  \\caption{Interface d'administration d'un site sous \\typo}
\\end{figure}

\\begin{figure}[h]
  \\centering
  \\includegraphics[width=0.5\\linewidth]{include/rte.png}
  \\caption{Éditeur de texte intégré à \\typo}
\\end{figure}

\\subsection{Qui utilise \\typo?}
Avec \\textbf{plus de \\nombre{9000}~références} en
ligne\\footnote{\\url{http://typo3.org/about/sites-made-with-typo3/}}
et plus de \\nombre{200000}~serveurs référencés, \\typo est la référence
de l'internet professionnel.

Parmi ces références figurent les sites Web
d'entreprises prestigieuses comme \\textit{Dassault Systèmes PLM
  Website}\\footnote{\\url{http://typo3.com/index.php?id=1386}},
\\textit{Philips}, \\textit{EDS}, \\textit{Volkswagen}, \\textit{General
  Electric}, \\textit{Stanford University}, \\textit{MGM Home
  Entertainment}, \\textit{3M}, \\textit{New York Times},
\\textit{Lufthansa}, \\textit{Ford}, \\textit{T-Online and Samsung}.

\\typo est aussi \\textbf{le \\og{}CMS\\fg{} de référence choisi par le
  gouvernement du Québec} avec 25 ministères et organismes québécois qui l'ont
choisi\\footnote{\\url{http://blogue.infoglobe.ca/2008/12/23/typo3-la-reference-au-gouvernement-du-quebec/}}
pour leur site internet, Intranet ou Extranet.

\\subsection{Pourquoi \\typo?}
Quand vient le moment de choisir un système de gestion de
contenu, trouver celui qui vous convient le mieux est \\textbf{un gage de
  pérennité et de plus-value}. Il est donc nécessaire de trouver celui
qui offrira \\textbf{la meilleure intégration} à votre structure et dont la
technologie et les fonctionnalités répondent à \\textbf{vos besoins
  actuels et futurs}.\\\\
Voici quelques faits qui font la différence:

\\begin{itemize}
\\item \\textbf{Richesse} de fonctionnalités;
\\item \\textbf{Facilité} d'utilisation;
\\item \\textbf{Prix} de la licence: 0€;
\\item \\textbf{Liberté} de modifier, étendre et adapter;
\\item \\textbf{Très grande base de clients};
\\item \\textbf{Coût de possession} (TCO) et \\textbf{Retour sur investissement} (ROI);
\\item \\textbf{Gestion de gabarits} (\\og{}Templates / Skins\\fg{}) puissante;
\\item \\textbf{Gestion des droits} très granulaire;
\\item \\textbf{Architecture robuste et évolutive};
\\item Plus de \\textbf{\\nombre{3000} extensions};
\\item \\textbf{Gestion multi-domaines/multi-sites};
\\item \\textbf{Respect des normes} et standards (\\textit{WAI et W3C}).
\\end{itemize}

\\subsection{Une communauté structurée et active}
La communauté mondiale associée à \\typo est très importante, très unie et très
organisée. Elle est tout aussi active en France, pour preuve le forum \\typo
français (\\url{http://forum.typo3.fr})  comprend près de \\nombre{4000} membres
enregistrés. Des \\og{}TUG\\fg{} (\\textit{\\typo User Group}) sont organisés sur toute la
France ainsi que des universités d’été (\\url{http://uni.typo3.fr/}) et
meetings dédiés à \\typo.
"))!§!

!§!(when  (yes-or-no-p "Insérer  proposition chiffrée ?")
(insert "\\section{Notre proposition chiffrée}
Notre offre comprend:
\\begin{itemize}
\\item l'écriture des spécifications fonctionnelles et techniques qui
  font office de contrat fonctionnelle et technique;
\\item détermination d'une charte graphique;
\\item la conception, la réalisation et l'intégration de votre site
  internet au sein de la solution \\typo;
\\item la validation fonctionnelle et technique;
  % \\item la production d'une documentation utilisateur et sa
  %   formation;
\\item la formation d'un utilisateur sur site;
\\item une période de maintenance à distance;
\\item l'hébergement du site et la gestion d'un nom de domaine qui
  comprend de plus 5 adresses de courriel protégées par un
  \\textit{anti-spam} et un \\textit{anti-virus}.
\\end{itemize}
"))!§!

{\textbf{\Large Coup forfaitaire \red{estimatif} à réajuster suivant
    les besoins réels et après concertation avec un chargé de projet: 3000€}.}

!§!(when  (yes-or-no-p "Insérer Référence ?")
(insert "
\\section{Nos références dans ce domaine}
\\begin{minipage}{0.65\\linewidth}
  \\begin{itemize}
  \\item \\textbf{\\url{http://www.leguideregional.fr}}.
    \\begin{description}
    \\item[Description:] Guide touristique et annuaire d'entreprises.
    \\item[Technologie utilisée:] \\typo.
    \\item[Remarque:] Site en cours de développement.
    \\end{description}
  \\end{itemize}
\\end{minipage}
\\begin{minipage}{0.33\\linewidth}
  \\centering
  \\includegraphics[width=3.5cm]{include/gr.png}
\\end{minipage}

\\vspace*{3mm}\\par
\\begin{minipage}{0.65\\linewidth}
  \\begin{itemize}
  \\item \\textbf{\\url{http://www.mairie-frouzins.fr/}}
    \\begin{description}
    \\item[Description:] Site officiel de la mairie de la ville de
      \\textsc{Frouzins} (plus de \\nombre{7000} habitants).
    \\item[Technologie utilisée:] \\typo.
    \\item[Remarque:] Développé en collaboration avec la société~INOVALYS.
    \\end{description}
  \\end{itemize}
\\end{minipage}
\\begin{minipage}{0.33\\linewidth}
  \\centering
  \\includegraphics[width=3.5cm]{include/frouzins.png}
\\end{minipage}

\\vspace*{3mm}\\par
\\begin{minipage}{0.65\\linewidth}
  \\begin{itemize}
  \\item \\textbf{\\url{http://promodecouverte.fr}}
    \\begin{description}
    \\item[Description:] Annuaire d'entreprises.
    \\item[Technologie utilisée:] \\typo.
    \\item[Remarque:] Le client est entièrement maître de la mise en
      forme du site.
    \\end{description}
  \\end{itemize}
\\end{minipage}
\\begin{minipage}{0.33\\linewidth}
  \\centering
  \\includegraphics[width=3.5cm]{include/promodecouverte.png}
\\end{minipage}

\\vspace*{3mm}\\par
\\begin{minipage}{0.65\\linewidth}
  \\begin{itemize}
  \\item \\textbf{\\url{http://www.piprime.fr}}.
    \\begin{description}
    \\item[Description:] Blog professionnel.
    \\item[Technologie utilisée:] \\texttt{WordPress}.
    \\item[Remarque:] Plus de \\nombre{800} articles référencés, plus de
      cent visiteurs différents pas jour.
    \\end{description}
  \\end{itemize}
\\end{minipage}
\\begin{minipage}{0.33\\linewidth}
  \\centering
  \\includegraphics[width=3.5cm]{include/piprime.png}
\\end{minipage}

\\vspace*{3mm}\\par
\\begin{minipage}{0.65\\linewidth}
  \\begin{itemize}
  \\item \\textbf{\\url{http://realisations.piprime.com/chrisfer/}}
    \\begin{description}
    \\item[Description:] Site d'artisan.
    \\item[Technologie utilisée:] PHP.
    \\item[Remarque:] Développé en collaboration avec la société~INOVALYS.
    \\end{description}
  \\end{itemize}
\\end{minipage}
\\begin{minipage}{0.33\\linewidth}
  \\centering
  \\includegraphics[width=3.5cm]{include/chrisfer.png}
\\end{minipage}


\\vspace*{3mm}\\par
\\begin{minipage}{0.65\\linewidth}
  \\begin{itemize}
  \\item \\textbf{\\url{http://realisations.piprime.com/rbchauffage/}}
    \\begin{description}
    \\item[Description:] Site d'artisan.
    \\item[Technologie utilisée:] PHP.
    \\item[Remarque:] Développé en collaboration avec la société~INOVALYS.
    \\end{description}
  \\end{itemize}
\\end{minipage}
\\begin{minipage}{0.33\\linewidth}
  \\centering
  \\includegraphics[width=3.5cm]{include/rbc.png}
\\end{minipage}


\\vspace*{3mm}\\par
\\begin{minipage}{0.65\\linewidth}
  \\begin{itemize}
  \\item \\textbf{\\url{http://realisations.piprime.com/rosedessables/}}
    \\begin{description}
    \\item[Description:] Site de commerçant.
    \\item[Technologie utilisée:] PHP.
    \\item[Remarque:] Développé en collaboration avec la société~INOVALYS.
    \\end{description}
  \\end{itemize}
\\end{minipage}
\\begin{minipage}{0.33\\linewidth}
  \\centering
  \\includegraphics[width=3.5cm]{include/rose.png}
\\end{minipage}
"))!§!
\end{document}