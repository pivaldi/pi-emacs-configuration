% From /usr/share/doc/tetex-frogg-doc/lettre/examples/lettre.tpl
\documentclass[11pt,%
twoside,%
leqno,%
fleqn,%
francais,%
origdate]{lettre}
\usepackage[utf8]{inputenc}
\usepackage[T1]{fontenc}
\usepackage{lmodern}
\usepackage[francais]{babel}
%
% Production d'etiquettes[commencant a la nieme etiquette de la page]
% ===================================================================
%
% \makelabels[1]

% Macro pour supprimer le trait de pliage
\makeatletter
\newcommand*{\NoRule}{\renewcommand*{\rule@length}{0}}
\makeatother

\begin{document}
%
% Declaration du fichier de defauts
% =================================
%
% Permet d'ecrire des lettres personalisees
% sans repreciser a chaque fois les parametres de l'expediteur
%
% \institut{fichier}
%
% Declaration du destinataire et environnement
% ============================================
%
% Permet d'ecrire plusieurs lettres a des destinataires differents
% sans repreciser les parametres de l'expediteur
%
\begin{letter}{Rectorat de Montpellier\\
    Division des personnels enseignants
    31, rue de l'université\\
    CS 39004\\
    34064 Montpellier
  }
  % Supprimer le trait de pliage
  % \NoRule
  %
  %
  % Parametre obligatoire
  % =====================
  %
  \name{Philippe Ivaldi}
  %
  % Parametres facultatifs de l'entete  % (defauts)
  % ===============================================
  %
  \address{Les Castaniès\\
    11250 PREIXAN%
  }
  % \psobs                               % ( Logo de l'Observatoire )
  % \location{Precision d'adresse}       % (                        )
  \telephone{06 25 43 88 07}   % (    +41(22) 755 26 11   )
  % \notelephone
  % \fax{No de fax expediteur}           % (    +41(22) 755 39 83   )
  \nofax
  \email{p22@ivaldi.xyx}       % (                        )
  %
  \lieu{Carcassonne}                    % (Sauverny, )
  % \nolieu
  % \date{date fixe}                     % (date courante)
  % \nodate
  %
  % Parametre de mise en page           % (defauts)
  % ==============================================
  %
  \marge{15mm}                          % (15mm)
  % \tension{1}                          % (2)
  %
  % Parametres facultatifs              % (defauts)
  % ===============================================
  %
  % \pagestyle{empty|headings}           % ( plain par defaut )
  % \francais|\romand|\anglais|          % (\francais)
  \francais
  % \americain|\allemand                 %/
  %
  \signature{Philippe Ivaldi\\
    \includegraphics{/home/pi/Documents/paperasse/administratif/signature.png}%
  }
  % \secondsignature{signature}          % ()
  % \thirdsignature{signature}           % ()
  %
  % \nref{reference}|\Nref{reference}    % ()
  % \vref{reference}|\Vref{reference}    % ()
  % \telex{numero}                       % ()
  % \ccp{numero}                         % ()
  % \faxobs                              % (+41(22) 755 39 83)
  % \ccpobs                              % (12-2130-4)
  %
  % \basdepage{texte}                    % ()
  % \username{nom d'utilisateur}         % ()
  % \internet{adresse RFC 822}           % ()
  % \ccitt{adresse X400}                 % ()
  % \bitnet{adresse bitnet}              % ()
  % \telepac{numero telepac}             % ()
  % \decnet{numero decnet}               % ()
  % \internetobs                         % ([username@]scsun.unige.ch)
  % \ccittobs                            % ([S=username;]OU=scsun;O=unige;%
  %                                     % PRMD=switch;ADMD=arcom;C=ch)
  %
  \conc{En réponse à xxxxxx}  % ()
  %
  % Corps de la lettre
  % ==================
  %
  \opening{Monsieur,}
  %
  CORPS DE LA LETTRE.
  %
  \closing{Je vous remercie de la bienveillance que vous voudrez accorder à ma requête et vous prie
    d'agréer, Monsieur, l'expression de mon plus profond respect.}
  %
  % Paragraphes supplementaires
  % ===========================
  %
  % \ps{label}{texte du post-scriptum}
  % \encl{CV.}
  % \cc{destinataires de copies conformes separes par des \\}
  %
\end{letter}
%
\end{document}
